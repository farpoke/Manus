\documentclass[sammanfattningar]{manus}

\usepackage[margin=3cm]{geometry}
\usepackage{lipsum}

\nyroll[red]{Foo}
\nyroll[green!70!black]{Bar}
\nyroll[blue]{Baz}

\begin{document}

\tableofcontents

\newpage
\section*{Bakgrund}
\addcontentsline{toc}{section}{Bakgrund}
\rentext{
\newcommand{\inhar}{\indent}
\parindent=1em
Vi är i mellankrigstidens Europa.
Ut från arkeologiska institutionen kommer en nyexaminerad man gående. Han är ung. Han är stolt. Han är norsk! Dessutom är han kär. \\
\inhar Med långa kliv tar han sig fram till en studentbostad. Ingen är hemma. \\
\inhar\citat{Hon är på museet, till ex-jobbet, du vet} säger en dansk student innan denne försvinner in i köket. Vår hjälte försvinner snabbt  mot museet. \\
%\\
\inhar I ett litet mörkt rum sitter en ung studentska lutad över ett dokument. Vår hjälte sväljer saliven och ler norskt. Sluttentan var ingenting jämfört med det här provet. Han faller ner på knä och blottar en ask. I asken finns ett ljuvligt halsband med en vackert utsmyckad båtberlock. \\
\inhar\citat{Ehm, Maria jeg...}\\
\inhar\citat{Thor! Vad gör du här!}, Maria reser sig snabbt och river ner det översta arket av flera hundra från ett hög med papper. Thor försöker hjälpa henne att fånga arket.\\
\inhar\citat{Jeg kom rett hit fra avslutningsseremonien. Jeg har tenkt veldig høyt av deg og ...}\\
\inhar\citat{Avslutningscer...} Marias ögon blixtrar till.\\
\inhar\citat{Ja, og nå som jeg har mer tid jeg ønsker å tilbringe den...}\\
\inhar\citat{Jag har inte tid med dig. Jag åker till sydamerika ikväll, vi får prata sen.}\\
\inhar Thor ser ner på försättsbladet  till Marias förstudie, nästa gång han tittar upp har hon redan hunnit flera korridorer bort. Asken ligger oanvänd i handen. Som ett rus kommer han på det. Klart att Maria var så brysk, han hade ju inte ens en riktig ring att erbjuda henne. \\
\inhar Glatt nynnande ger sig Thor Heyerdahl ut på på äventyret i jakt på förlovningsringen\footnote{\normalsize Släpps snart till en bokhandel/biograf/godisbutik nära dig.}.\\
\inhar Maria i sin tur ger sig iväg till ett mycket speciellt tempel som Mayacivilisationen lämnat efter sig.
}

\newpage


 \begin{rollbeskrivning}
\namn{Wicky} 
 \bild{wicky}
 \beskrivning{En kvinna i karriären. Startade redan som treåring ett saftstånd och konkurrerade ut de andra ungarna på gatan med hjälp av maf… avancerad marknadsföring. Vid tjugo års ålder så fick hon anställning vid REMC, Royal English Make-up Corporation, och befodrades väldigt snart till försäljningschef. Olyckligtvis så kom militären på att kriget var över och då sågs smuggling av smink inte längre som något man kunde tolerera. Med andra ord så fick man rebranda och med känslan hos en svt-chef så bytta man helt sonika namn till REMC, Really Evil Medical Corporation, och började sälja läkemedel.  }
 \end{rollbeskrivning}
 
 \begin{rollbeskrivning}
\namn{Arthur} 
 \bild{arthur}
 \beskrivning{ En britt i sina bästa år. Arthur började sin vetenskapliga bana med att läsa organkemi vid University of Warwick. Efter detta så fortsatte han med att doktorera vid Brasneck college. Doktorsstudierna ledde sedan till att han fick en anställning vid REMC. Ett hederligt brittiskt företag involverade i sådana fina traditioner som kolonialismn och protektionism. Det känns tryggt att vara involverad i något så trevligt. I synnerhet när det brinner lite i knutarna.}
 \end{rollbeskrivning}

  \begin{rollbeskrivning}
\namn{Wicky} 
 \bild{mumien}
 \beskrivning{En kvinna i karriären. Startade redan som treåring ett saftstånd och konkurrerade ut de andra ungarna på gatan med hjälp av maf… avancerad marknadsföring. Vid tjugo års ålder så fick hon anställning vid REMC, Royal English Make-up Corporation, och befodrades väldigt snart till försäljningschef. Olyckligtvis så kom militären på att kriget var över och då sågs smuggling av smink inte längre som något man kunde tolerera. Med andra ord så fick man rebranda och med känslan hos en svt-chef så bytta man helt sonika namn till REMC, Really Evil Medical Corporation, och började sälja läkemedel.  }
 \end{rollbeskrivning}

 \begin{rollbeskrivning}
\namn{Arthur} 
 \bild{zots}
 \beskrivning{ En britt i sina bästa år. Arthur började sin vetenskapliga bana med att läsa organkemi vid University of Warwick. Efter detta så fortsatte han med att doktorera vid Brasneck college. Doktorsstudierna ledde sedan till att han fick en anställning vid REMC. Ett hederligt brittiskt företag involverade i sådana fina traditioner som kolonialismn och protektionism. Det känns tryggt att vara involverad i något så trevligt. I synnerhet när det brinner lite i knutarna.}
 \end{rollbeskrivning}
 


\akt

\scen{Har har har har...}{Någonting händer!}

\berattare{Ljuset släcks och fram genom ridån kikar årets producent och årets matgruppschef. Dessa presenterar spexet och årets mat. Ivriga spexare serverar därefter den underbara förrätten. När publiken har ätit sig belåtna återkommer producenten och ropar in Bandet! Efter Bandet levererat ett makalöst aktintro stiger applåder från publiken.}

\Fooin
\Barin
\Foo{Bar}
\Bar{Lorem ipsum...}
\berattare{Something is said.}
\Foogr{Aha! I've done it, using \rekv{this}!}
\Foo{\"Byr}
\Foout
\Bazin
\Baz{Bar Fooy! and stuff. \gr{Lorem ipsum dolor sit amet.} And so on and so on... Foo. Bar!}

\begin{kuplett}
	\titel{Livselixir och evighetens tempel}
	\deltagare{Maria, Amaya, Wicky och Arthur}
	\beskrivning{
		Maria berättar om sin exjobbsförstudie och templet och dess historia. Fokus på Maria.
	}
	\melodi{Dr Jones}
\end{kuplett}

\Foo{Test?}
\Bazsub
\Bar{Hej, nu går jag in i subspel.}
\Barsub
\Fooin
\Foo{Hello again!}
\allgr{Alla gör något.}
\Foo{\lipsum[1]\vspace{-\baselineskip}}
\Barin
\indikatorbredd{1pt}
\Baz{Haha! Jag ändrar \rekv{indikatorbredden!}}
\indikatorbredd{1em}
\Barsubslut
\Bar{And... back again!}
\Barut
\Bazut

\scen{What?}{How?}

\Foo{Hello again!}
\Bazin
\Baz{Lorem ipsum dolor sit amet, consectetuer adipiscing elit. Ut purus elit, vestibulum ut, placerat ac, adipiscing vitae, felis. Curabitur dictum gravida mauris. Nam arcu libero, nonummy eget, consectetuer id, vulputate a, magna. Donec vehicula augue eu neque. Pel- lentesque habitant morbi tristique senectus et netus et malesuada fames ac turpis egestas. Mauris ut leo. Cras viverra metus rhoncus sem.}

\begin{kuplett}
	\titel{Göra-hemsk-inhemsk-ritual}
	\beskrivning{
		Ritual där man blandar en massa konstiga saker i en \rekv{kittel}. En \rekv{amulett} samt en lagom mängd \rekv{pappingredienser} figurerar. Även \rekv{Duvan Bengt} är en viktig ingrediens. Mycket fokus på Zots.
	}
	\deltagare{Thor, Mumien, Zots}
	\melodi{Pudding med arsenik, Welcome to the 60s}
\end{kuplett}

\scen{}{
	Hello! 1
}
\scen{}{
	Hello! 2
}
\scen{}{
	Hello! 3
}

\statistiktabell

\printindex

\sammanfattningslista

\end{document}
