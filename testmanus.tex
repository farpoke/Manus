\documentclass[sammanfattningar]{manus}

\usepackage[margin=3cm]{geometry}
\usepackage{lipsum}

\nyroll[red]{Foo}
\nyroll[green!70!black]{Bar}
\nyroll{Hoy}
\nyroll[blue]{Baz}

\begin{document}

\tableofcontents

\newpage

\akt

\scen{Har har har har...}{Någonting händer!}

\berattare{Ljuset släcks och fram genom ridån kikar årets producent och årets matgruppschef. Dessa presenterar spexet och årets mat. Ivriga spexare serverar därefter den underbara förrätten. När publiken har ätit sig belåtna återkommer producenten och ropar in Bandet! Efter Bandet levererat ett makalöst aktintro stiger applåder från publiken.}

\Fooin
\Barin*
\Foo{Bar}
\Bar{Lorem ipsum...}
\berattare{Something is said.}
\Foogr{Aha! I've done it, using \rekv{this}!}
\Foo{\"Byr}
\Foout
\Bazin
\Baz{Bar Fooy! and stuff. \gr{Lorem ipsum dolor sit amet.} And so on and so on... Foo. Bar!}

\begin{kuplett}
	\titel{Livselixir och evighetens tempel}
	\deltagare{Maria, Amaya, Wicky och Arthur}
	\beskrivning{
		Maria berättar om sin exjobbsförstudie och templet och dess historia. Fokus på Maria.
	}
	\melodi{Dr Jones}
\end{kuplett}

\Foo{Test?}
\Bazsub
\Bar{Hej, nu går jag in i subspel.}
\Barsub
\Fooin
\Foo{Hello again!}
\allgr{Alla gör något.}
\Foo{\lipsum[1]\vspace{-\baselineskip}}
\Barin
\indikatorbredd{1pt}
\Baz{Haha! Jag ändrar \rekv{indikatorbredden!}}
\indikatorbredd{1em}
\Barsubslut
\Bar{And... back again!}
\Barut
\Bazut

\scen{What?}{How?}

\berattare{Här nämner berättaren lite \rekv{rekvisita}.}
\Foo{Hello again!}
\Bazin
\Baz{Lorem ipsum dolor sit amet, consectetuer adipiscing elit. Ut purus elit, vestibulum ut, placerat ac, adipiscing vitae, felis. Curabitur dictum gravida mauris. Nam arcu libero, nonummy eget, consectetuer id, vulputate a, magna. Donec vehicula augue eu neque. Pel- lentesque habitant morbi tristique senectus et netus et malesuada fames ac turpis egestas. Mauris ut leo. Cras viverra metus rhoncus sem.}

\begin{kuplett}
	\titel{En annan kuplett}
	\beskrivning{Allmänt flumm.}
\end{kuplett}

\rekvisitalista
\kuplettlista
\sammanfattningslista
\statistiktabell

\end{document}
