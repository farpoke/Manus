\documentclass[a4paper,12pt]{article}

\usepackage{ifxetex}
\ifxetex
	\RequirePackage[T1]{fontenc}
\else
	\RequirePackage[utf8]{inputenc}
\fi

\usepackage[pdftex, margin=3cm]{geometry}

\usepackage{xifthen}
\usepackage{url}
\usepackage[bottom,hang]{footmisc}
\usepackage[babel]{csquotes}
\usepackage[british,swedish]{babel}
\usepackage[colorlinks]{hyperref}
\usepackage{datetime}
\usepackage[dvipsnames]{xcolor}

\usepackage{fancyvrb}
\DefineShortVerb{\|}

\usepackage{listings}
\lstset{
	basicstyle=\small\ttfamily,
	columns=flexible,
	numberstyle=\tiny,
	language=TeX,
	keywordstyle=\bfseries,
	morekeywords={documentclass,definecolor,scenanvisning,sekvens,flöde,gå,om,namn,bild,scenanv,berattare,nygrupp,Namngrupp,nyroll,nyhjalproll,aktiveraNamn,indikator,rekvisita,ljud,efkt,ljudeffekt,Namnsj,Namngör,Namnregi,deaktiveraNamn,begin,gr,rekv,akt,scen,allgr,nyhjälproll,titel,deltagare,beskrivning,melodi,Namn,Namngr,Namnin,Namnut,Namnsub,Namnsubslut,rekvisitalista,kuplettlista,sammanfattningslista,statistiktabell,namnbyte,fotfil,pagestyle,thispagestyle,textit,fotstil,fotstil@yttre,textcolor,rule,fothar}
}

\renewcommand\ttdefault{pcr}

\usepackage{makeidx}
\def\at{@}
\makeindex

\usepackage{needspace}

\setcounter{secnumdepth}{1} 

\newcommand*{\pack}{\textsf}

\renewcommand{\dateseparator}{-}
\newcommand{\todayiso}{%
	\the\year\dateseparator\twodigit\month\dateseparator\twodigit\day}

\newcommand\funnybone[1]{%
	\raisebox{1em}[1em][0pt]{%
		\makebox[\textwidth][r]{\large\it eller, ``#1''}%
	}%
}



\begin{document}

\title{Att skriva manus med \pack{manus.cls}}
\author{%
	Cecilia Kjellman, \texttt{cecilia.kjellman@gmail.com}\\%
	Artemis Mårtensson, \texttt{artemis.martensson@gmail.com}}
\date{\todayiso}
\maketitle



\newpage
\tableofcontents
\newpage



\section{Vad är \pack{manus.cls}?}

\subsection{Kort svar}
En dokumentklass till \LaTeX\ som är skriven med manus till scenframträdanden i tankarna.

\subsection{Något längre svar}
Klassen ger tillgång till ett antal kommandon som tillsammans ger möjlighet att skriva ett snyggt och lättläst manus. Vissa av dessa ger möjlighet till sådana flashiga funktioner som att skriva färgkodade repliker, indikatorer på vilka som är på scen, listor över rekvisita/kupletter/sammanfattningar, numrerade repliker och statistik för respektive roll. Andra är mer av estetisk natur, så som |\akt|. 

\subsection{Kan jag använda klassen trots att jag inte kan \LaTeX?}
Du behöver veta hur man kompilerar filer samt förstå syntaxen som presenteras i detta dokument. Ifall något skulle gå sönder så underlättar det om du har grundläggande felsökningserfarenhet.



\section{Från början: Klassinställningar och preamble}
Låt oss börja med en kort exempel-preamble:

\begin{lstlisting}
\documentclass{manus}
	% <-- Ev. laddning av fler paket.
	% Så som inputenc eller fontenc.
\definecolor{rosa}{rgb}{1,0.5,0.5}
\nyroll[rosa]{Bobby}
\end{lstlisting}

\noindent
Nu är du redo att skriva ett manus om Bobby som kommer få alla sina repliker i rosa! Men låt oss först titta närmare på vad som faktiskt står ovan.

\subsection{Ladda klassen}
\index{documentclass@{\verb#\documentclass{manus}#}}
På rad 1 ovan så anges att önskad dokumentklass är \pack{manus}, vilken bygger på dokumentklassen \pack{article}.

För att ge direktiv till manusklassen kan extra klassalternativ anges på följande sätt:

\begin{lstlisting}
\documentclass[alternativlista]{manus}
\end{lstlisting}

\noindent
där |alternativlista| är en komma-separerad lista med alternativ (eller tom för den delen).

\subsubsection{Alternativ: \texttt{svartvit}}
\label{alt:svartvit}
\index{svartvit@\texttt{svartvit}}
Som standard är alla repliker mm. skrivna med färg. Detta alternativ sätter alla textfärger till svart. Tänkt användningsområde är främst om man vill skriva ut i svartvitt utan att ändra färger manuellt. 

\subsubsection{Alternativ: \texttt{sammanfattningar}}
\label{alt:sammanfattningar}
\index{sammanfattningar@\texttt{sammanfattningar}}
När ny scen påbörjas med kommandot |\scen| (se nedan) kan en sammanfattning av scenen anges. I normala fall sparas den bara undan för att skrivas ut i slutet av dokumentet i sammanfattningslistan (se nedan). Men om detta klassalternativ ges så skrivs sammanfattningen även ut direkt efter scenrubriken.



\subsubsection{Alternativ: \texttt{todo}}
\label{alt:todo}
\index{todo@\texttt{todo}}
Detta alternativ gör att påminnelser skrivna med kommandot |\todo{...}| visas i texten. En lista över kommentarer kan inkluderas med kommandot |\todolista|. Listan fungerar även om |todo|-alternativet inte anges. Tänkt anvädningsområde är att kunna skriva in påminnelser om sådant som finns kvar att göra i direkt anslutning till där det skall göras i texten.

\subsubsection{Alternativ: \texttt{kommentarer}}
\label{alt:kommentarer}
\index{kommentarer@\texttt{kommentarer}}
Gör liknande sak som |todo| men med kommandot |\kommentar|, samt att utseendet är något annorlunda. En lista över kommentarer kan inkluderas med \\|\kommentarlista|. Tänkt användningsområde är att man kan ha en kommentar/tydliggörande som står kvar även när man gömt alla att-göra/todo-markeringar.

\subsubsection{Alternativ: \texttt{kallrad}}
\label{alt:kallrad}
\index{kallrad@\texttt{kallrad}}
Gör att källradsnumret (dvs den rad i |.tex|-filen där repliken är skriven) visas till vänster om varje repliknumer.


\subsubsection{Alternativ: \texttt{grupper}}
\label{alt:grupper}
\index{grupper@\texttt{grupper}}
Detta alternativ aktiverar en special-stapel i högermarginalen för grupper. Den skapades för musikalen Spelarna, men blev i slutändan inte använd. Alternativet finns kvar utifall det blir användbart i framtiden.


\subsubsection{Alternativ: \texttt{grafer}}
\label{alt:grafer}
\index{grafer@\texttt{grafer}}
Detta alternativ gör att statistiktabellen (se avsnitt \ref{sec:statistik}) följs av diverse illustrativa grafer.


\subsection{Definiera karaktärer/roller}
\label{sec:nyroll}
\index{Definiera karaktärer}
\index{nyroll@\verb#\nyroll#}
\index{nyhjalproll@\verb#\nyhjalproll#}
För att kunna använda manusfunktionaliteten behöver man definera vilka karaktärer som finns. Detta görs med två olika kommandon:

\begin{lstlisting}
\nyroll[färg]{namn}      % Ny karaktär.
\nyhjalproll[färg]{namn} % Ny hjälpkaraktär.
\nyhjälproll[färg]{namn} % Ny hjälpkaraktär.
\end{lstlisting}

De första varianten definierar en ny karaktär och ger även denna en närvaroindikator som ritas ut i högermarginalen. Mer om denna indikator i avsnitt \ref{sec:narvaro}.

Den andra varianten definierar också en ny karaktär, men skapar \emph{inte} en närvaroindikator eller de kommandon som associeras med en sådan. Detta är användbart för exempelvis småroller som bara finns med i en scen.

För ett mer avancerat sätt att definiera karaktärer, se \ref{sec:nyrollII}.

\subsubsection{Färger}
\label{sec:farger}
\index{Färger}
Parametern |[färg]| är frivillig och om den inkluderas kommer den givna färgen användas för den karaktärens repliker, agerande samt för karaktärens närvaroindikator och statistik.

Detta är som sagt frivilligt, och svart kommer tilldelas som karaktärsfärg om inget annat anges. 

Värdet på färgparametern skall vara en färg enligt paketet \pack{xcolor}, som är det paket som \pack{manus.cls} använder för färgläggning. Färgerna som listas i appendix \ref{app:farger} borde räcka, men man kan även definiera egna färger eller ange en blandning av färger enligt \pack{xcolor}:s format. För detaljer se \pack{xcolor}:s dokumentation (googla ``Package xcolor'') .



\needspace{5em}
\section{Manus}
Början på ett exempelmanus:

\begin{lstlisting}
\begin{document}
	% <-- Ev. andra sidor så som titelsida, innehåll, etc.
\akt
\scen{Bobby pratar}{
	Bobby pratar lite, gör något och pratar lite till.
}
\Bobby{Detta säger jag, och se här \gör{gör något} vad jag gör.}
\Bobbygör{Bobby gör något.}
\Bobby{Här säger jag något igen.}
\end{lstlisting}



\subsection{Akter och scener}
\index{akt@\verb#\akt#}
\index{scen@\verb#\scen#}
Ett manus kan delas in i akter och scener (för mer se \ref{sec:sekvenser}), och dessa inleds med följande två kommandon:

\begin{lstlisting}
\akt[Första scennummer]
\scen{Titel}{Sammanfattning}
\end{lstlisting}

\paragraph{Första scennummer} är frivilligt att ange och anger var numreringen av scenerna skall börja. Detta är användbart eftersom vissa akter har en scen 0, andra inte. Om detta inte anges antas att numreringen skall börja på 1.

\paragraph{Titel} är titel för scenen.

\paragraph{Sammanfattning} är sammanfattning för scenen. Denna används för att generera sammanfattning utav hela manuset (se avsnitt \ref{sec:slut-på-manus}). Om klassalternativet \texttt{samman\-fattningar} har angetts skrivs sammanfattningen även ut direkt efter scentiteln. Användningsområde har varit att skriva ett synopsismanus med scensammanfattningarna och sedan kunna se och använda scensammanfattningarna vid skrivandet av scenen. För att manusklassen skall fortsatt vara kompatibel med gamla manus så är sammanfattningen fortfarande obligatorisk men man kan såklart lämna den tom.




\subsection{Repliker och regi}
\label{sec:repliker}
\index{Repliker}
\index{Agerande}
Den allmänna formen för att skriva repliker är:
\index{Namn@\verb#\Namn#}
\index{Namngr@\verb#\Namngr#}
\index{Namnregi@\verb#\Namnregi#}
\begin{lstlisting}
\Namn{...}
\Namnsj{...}
\Namngr{...}
\Namngör{...}
\Namnregi{...}
\end{lstlisting}
där ``Namn'' är tidigare definierad i |\nyroll| (se \ref{sec:nyroll}) och där |\Namn| skriver ut en vanlig replik från denne karaktären, |\Namnsj| skriver ut en replik markerad som sång, |\Namngr|,|\Namngör|,|\Namnregi| skriver ut en rad markerad som agerande.

Finessen med |\Namngr| är att kunna skriva längre mängd text för att beskriva att någon gör något. För att markera samma sak mitt i en replik används |\gr|.

Då flera repliker i rad är från samma karaktär kommer namnet endast skrivas ut för den första raden.



\subsection{Scennärvaro}
\label{sec:narvaro}
\index{Namnin@\verb#\Namnin#}
\index{Namnut@\verb#\Namnut#}
De karaktärer som definierats med kommandot |\nyroll| får en närvaroindikator i högermarginalen som visar om karaktären är på scen eller inte samt om denna är i subspel eller inte.

Denna indikator kontrolleras med följande kommandon:
\begin{lstlisting}
\Namnin
\Namnut
\Namnsub
\Namnsubslut
\deaktiveraNamn
\aktiveraNamn
\indikator{etikett}{färg}
\end{lstlisting}

Till exempel: när |\Bobbyin| skrivs kommer Bobby markeras som inne på scen och närvaroindikatorn för Bobby kommer visas i marginalen. När Bobby sedan skall av scen skriver man |\Bobbyut| och närvaroindikatorn avslutas.

Om man vill markera subspel så kan man använda |\Bobbysub|, vilket gör att Bobbys närvaroindikator blir streckad. När Bobby är klar med vad han än höll på med och återigen skall agera stort skrivs |\Bobbysubslut|.

|\Namnin|- och |\Namnut|-kommandona skriver automatiskt ut regi som meddelar att karaktären i fråga ``kommer in på scen'' respektive ``går av scen.'' Då detta ej önskas läggs en asterisk till efter kommandot (till exempel |\Bobbyin*|).

Det spelar ingen roll om man inte skriver |\Namnsubslut| innan |\Namnut| (dvs karaktären i fråga går av scen direkt från subspel) men |\Namnin| kommer automatiskt ta karaktären ut ur subspel.

Vill man inte ha indiktorn alls, exempelvis om man inte vill att att den tar upp plats i marginalen, så kan man deaktivera den: |\deaktiveraBobby|. Motsatsen görs med |\deaktiveraBobby|. Väljs inget är indikatorn aktiverad.

Vill man ha en indiktor utan att skapa en karaktär kan man göra detta med hjälp av kommandot |\indikator{Indnamn}{färg}|. Då kan man använda kommandona ovan med Indnamn utan att ha skapat en karaktär.



\subsection{Sång och musik}
\paragraph{Sång}
\index{Sång}
Om man vill skriva repliker som markeras som sångtext kan detta göras med varje respektive karaktärs |\Namnsj|-kommando.

\paragraph{Kuplett}
\index{Kuplett}
Vill man markera kupletter i manus kan man göra det med:
\begin{lstlisting}
\begin{kuplett}[kommandon]
	\titel{...}
	\beskrivning{...}
	\melodi{...}
\end{kuplett}
\end{lstlisting}
där samtliga kuplettkommandon (|\titel| osv...) är frivilliga. Detta skriver ut en ny kuplett i manus och lägger även till samma kuplett i innehållsförteckningen och sammanfattningslistan.

Alla karaktärer som är på scen (inklusive subspel) när kupletten skrivs antas vara deltagare och deras namn kommer listas som sådana.

Parametern |[kommandon]| är en frivillig parameter som kan användas för att ange extra kommandon som skall gälla innuti kuplettmiljön. Detta är främst tänkt att användas för att enkelt justera vilka som skall vara på scen under kupletten.

Om till exempel Bobby och Eva är av scen men skall in på scen under en kuplett kan man specificera:
\needspace{3\baselineskip}
\begin{lstlisting}
\begin{kuplett}[\Bobbyin*\Evain*]
% ...
\end{kuplett}
\end{lstlisting}
Detta gör att Bobby och Eva kommer in under kupletten och markeras som deltagare, och sedan automatiskt går av scen efter kupletten.

\paragraph{Obs} Det finns ingen speciell begränsning på vad som får skrivas inuti kuplett\-kommandona, men det finns risk att andra delar av kuplettfunktionaliteten (speciellt sammanfattningarna) går sönder då underligare kommandon används inuti till exempel beskrivningen.

\paragraph{Musik}
Vill man ha samma funktionalitet som kuplett, men ogillar ordet 'kuplett' så kan man istället använda musik-miljön och få samma resultat fast med ordet musik.

\begin{lstlisting}
\begin{musik}[kommandon]
	\titel{...}
	\beskrivning{...}
	\melodi{...}
\end{musik}
\end{lstlisting}


\section{Övriga hjälpmedel}
För att underlätta manusskrivande finns även några extra kommandon:

\subsection{Scenanvisningar}
\begin{lstlisting}
\scenanvisning{...}
\scenanv{...}
\berattare{...}
\end{lstlisting}

\index{Berättare}
\index{berattare@\verb#\berattare#}
\index{Scenanvisning}
\index{scenanvisning@\verb#\scenanvisning#}
\index{scenanv@\verb#\scenanv#}
Dessa kommandon gör samma sak och finns för att berätta något som händer som inte angår någon karaktär direkt. Så som hur kulisserna ser ut eller vilken rekvisita som står på scen. Rekvisita som nämns här kommer markeras annorlunda i listan över rekvisita (se avsnitt \ref{sec:slut-på-manus}). Anledningen till att det finns flera sätt att anropa samma kommando är olika personer föredragit olika sätt att skriva kommandot och undertecknad har försökt göra alla glada. Notera även att detta inte är samma sak som att ha en berättar-roll med repliker.

\subsection{Rekvisita}
\begin{lstlisting}
\rekvisita{sak}
\rekv{sak}
\rekvisita[listsak]{sak}
\rekv[listsak]{sak}
\end{lstlisting}
\index{Rekvisita}
\index{rekvisita@\verb#\rekvisita#}
\index{rekv@\verb#\rekv#}
När rekvisita refereras till i manus kan den markeras med dessa kommandon som göra samma sak. De markerar rekvisita-texten och indexerar även densamma i sakregistret. Om man vill kan man ange olika namn på rekvisitan i texten respektive listan genom att ange hur det skall synas i listan. Anledningen till att det finns flera sätt att anropa samma kommando är olika personer föredragit olika sätt att skriva kommandot och undertecknad har försökt göra alla glada.


\subsection{Ljudeffekter}
\begin{lstlisting}
\ljud{ljudet}
\ljud[hur det listas]{ljudet}
\efkt{ljudet}
\efkt[hur det listas]{ljudet}
\ljudeffekt{ljudet}
\ljudeffekt[hur det listas]{ljudet}
\end{lstlisting}
Kan användas för att markera ljudeffekter i manus och listar det i ljudlistan. Precis som med rekvisitan kan man ange separat hur det listas om man så vill. Anledningen till att det finns flera sätt att anropa samma kommando är olika personer föredragit olika sätt att skriva kommandot och undertecknad har försökt göra alla glada.


\subsection{När det angår alla}
Ifall flera roller är involverade kan man skriva regi med |\allgr{}|\index{allgr@\verb#\allgr#}, och sång med |\allsj{}|\index{allsj@\verb#\allgr#}.
Detta får då en egen formatering i vänstermarginalen. 
\\
Tycker man inte om denna specialformatering så skulle undertecknad rekommendera att man definierar en hjälproll vid namn 'Alla' som då kan använda alla sedvanliga kommandon som hör ihop med roller.
Så här:
\begin{lstlisting}
\nyhjälproll{Alla}
\end{lstlisting}


\subsection{Övriga övriga...}
\paragraph{\tt\bfseries\textbackslash comment\{...\}}
\index{comment@\verb#\comment#}
Detta kommando finns ifall man utav någon anledning inte vill kommentera ut resten av en rad med det på \TeX\ sedvanliga sättet med \%. Detta kommando tar helt enkelt bara bort det som befinner sig inom måsvingarna, och kan göra koden mer svårläst.

\paragraph{\tt\bfseries\textbackslash TODO\{...\}}
\index{TODO}
\index{todo@\verb#\todo#}
Detta kommando lägger till en TODO-notering med angiven text. Då klassalternativet |todo| är angivet kommer denna notering skrivas ut direkt vid anrop, annars inte. I båda fallen läggs det till i TODO-listan som kan skrivas ut med kommandot |\todolista|.

\paragraph{\tt\bfseries\textbackslash kommentar\{...\}}
\index{kommentar}
\index{kommentar@\verb#\kommentar#}
Detta kommando lägger till en kommentar på liknande sätt som |\TODO| ovan, med skillnaden att klassalternativet |kommentarer| styr, och att listan över kommentarer ges med |\kommentarlista|.

\paragraph{\tt\bfseries\textbackslash citat\{...\}}
\index{citat}
\index{citat@\verb#\citat#}
Om man, likt undertecknad, har svårt att komma ihåg hur man sätter ut citat-tecken så kan man använda detta kommando för att omsluta något med citat-tecken. 

\paragraph{\tt\bfseries\textbackslash mellanrum\{...\}}
\index{mellanrum}
\index{mellanrum@\verb#\mellanrum#}
Kan användas för att skapa ett mellanrum i manus med indikatorer i högermarginalen. Ange mellanrummet med en siffra och enhet. Giltiga enheter: em, ex, in, pt, pc, cm, mm, dd, cc, bp, och sp. Exempel: |\mellanrum{3em}|.

Vanliga \LaTeX-kommandon för att få/ta bort extra utrymme, exempelvis |\vspace{3em}|, kan också användas.



\subsection{Definiera roller, episod II}
\index{Definiera karaktärer}
\index{nyroll@\verb#\nyroll#}
\index{nyhjalproll@\verb#\nyhjalproll#}
\label{sec:nyrollII}
I avsnitt \ref{sec:nyroll} beskrevs kommandon för att definiera karaktärer i manuset. Dessa var dock inte helt kompletta. 
De faktiska kommandona som definierar roller ser ut så här:

\begin{lstlisting}
\nyroll[färg]{namn-a}[namn-b]      % Ny karaktär.
\nyhjalproll[färg]{namn-a}[namn-b] % Ny hjälpkaraktär.
\end{lstlisting}

\noindent
där parametrarna |[färg]| och |{namn-a}| motsvarar de tidigare givna. 

Skillnaderna mellan |namn-a| och |namn-b| är att det första är det namn som används för kommandon, medans det andra är det namn som faktiskt skrivs ut i manus. |[namn-b]| är en frivillig parameter och om den utelämnas kommer |namn-a| användas.

Så, om vi till exempel vill ha en berättarröst som karaktär (inte samma sak som |\berattare| ovan) kan vi göra så här:

\begin{lstlisting}
\nyroll{Berattare}[Berättare]
% ... manus ...
\Berattare{Här säger berättaren någonting.}
\end{lstlisting}

\vspace{-1em}
\paragraph{Namnbyte} En annan sak som nämndes var namnbyte, som åstakomms med kommandot |\namnbyte{namn-a}{namn-b}| som ändrar det namn som skrivs ut (|namn-b|) för karaktären med kommandonamn |namn-a|. Tänkta användningsområden är att kunna byta namn på karaktärer som byter namn i pjäsen, eller om skådespelaren skall fortsätta i en ny karaktär med samma färg, indikator och statistik. Ett annat användningsområde är att kunna skriva något annat än namnet i källkoden men få hela namnet i det färdiga manuset, så som en förkortning eller slippa skriva ut ett svårstavat namn. Se bara till att alla författare är medveten om förkortningar/stavningsförenklingar.

Behöver man vid något tillfälle skriva ut namnet på karaktären via kommando (så det ändras ihop med namnbytet) så går det att göra genom att lägga till ändelsen |namn|, exempel: |\Bobbynamn|.

\subsection{Grupper}
Till musikalen Spelarna fanns behov av att markera om ljus/fokus skulle finnas på en eller båda av två scener. Därför skapades möjligheten att skapa grupper,  tilldela karaktärer grupper, och få markeringar som indikerar vilken/vilka grupper som skall ha ljus samt om någon replik sägs av karaktär som tillhör en grupp som inte har fokus. Undertecknad är tveksam att de kommer till användning igen, men utifall de gör det så finns de kvar.

\begin{lstlisting}
\nygrupp{gruppnamn}{färg}     % Ny grupp
\Namngrupp{gruppnamn} % Lägg till karaktär till grupp
\end{lstlisting}

\subsection{Sekvenser}
\label{sec:sekvenser}
För musikalen Apornas Ö delades Scener in i mindre bitar kallade Sekvenser. Skulle det bli användbart igen finns det kvar. Sekvenser numreras och kan refereras till med hjälp av angiven referens. I sekvenserna kan det finnas instruktioner om vilken sekvens att gå till härnäst.

\begin{lstlisting}
\sekvens{referens} % Sekvens som kan refereras till
\flöde{...} % Text som instruerar till byte av sekvens
\gå{referens} % Instruktion att byta till sekvensen med referensen
\om{villkor}{referens} % Instruerar att byta sekvens givet villkoret
\end{lstlisting}


\subsection{Sidfötter}
\index{Sidfötter}
\index{fotfil@\verb#\fotfil#}
Manusklassen innehåller ett kommando för att specificera en fil med sidfötter som skall användas:

\begin{lstlisting}
\fotfil[layoutnamn]{filnamn}
\end{lstlisting}

Detta kommando skapar en ny sidlayout som för varje sida läser in en rad från den angivna filen och använder den raden som sidfot för den aktuella sidan. Tomrader i filen resulterar i sidor utan sidfotstext (en sida per rad), dock fortfarande med sidnummer på rätt ställe.

Parametern |[layoutnamn]| är frivillig och namnet |fotfil| kommer användas om den utelämnas. Detta namn är det namn som ges den nya sidlayouten och ett fel kommer ges om en sidlayout med det namnet redan finns. Den nya sidlayouten aktiveras sedan direkt.

\paragraph{Layoutändring} Sidlayouter som skapas via detta kommando kan precis som \\\LaTeX:s vanliga sidlayouter styras via kommandona |\pagestyle| och \\|\thispagestyle|. Dvs om man vill tillfälligt ändra layout kan man till exempel använda kommandot
\begin{lstlisting}
\thispagestyle{plain}
\end{lstlisting}
som för just denna sidan ändrar layouten till |plain| (\LaTeX:s defaultlayout).

Om man vill helt gå tillbaka till vanligare layouter kan man exempelvis använda
\begin{lstlisting}
\pagestyle{plain}
\end{lstlisting}
som ändrar layouten för följande sidor. Om man senare vill ändra tillbaka kan man använda samma kommando men ange en layout som skapats via |\fotfil|. Om layoutnamnet inte angavs explicit så användes |fotfil| som namn, och kommandot för att ändra tillbaka blir då
\begin{lstlisting}[keywords={pagestyle}]
\pagestyle{fotfil}
\end{lstlisting}

Kommandot |\fotfil| kan dessutom användas flera gånger för att skapa olika sidlayouter som läser från olika filer, och man kan sedan fritt växla mellan dem med hjälp av ovan nämnda kommando.

\paragraph{Utseende}\index{fotstil@\verb#\fotstil#} Utseendet på sidfötterna styrs främst av kommandot |\fotstil|. Detta kommando kan defineras till att ta noll eller en parameter och används innan sidfotstexten skrivs ut.

Manusklassens definition av |\fotstil| ser ut så här:
\begin{lstlisting}
\let\fotfil\textit
\end{lstlisting}
vilket gör att sidfotstexten skrivs ut i kursiv stil.

Andra exempel är:
\begin{lstlisting}
\let\fotstil\bf % Fetstil.
\def\fotstil#1{\textcolor{green}{#1}} % Grön text.
\def\fotstil{\rule{5pt}{5pt}} % En liten låda innan texten.
\end{lstlisting}

\paragraph{Finkontroll}\index{forhär@\verb#\fothär#} I vissa fall kan det vara önskvärt att sätta en specifik fotnot på en sida. Istället för att pilla med fot-filen så att rätt sak hamnar på rätt sida finns istället kommandot

\begin{lstlisting}
\fothär{...}
\fothar{...}
\end{lstlisting}

Detta kommando gör att den angivna texten används som fotnot på den nuvarande sidan, istället för texten från fot-filen. Normala fotnötter kommer sedan fortsätta på nästa sida, såvida inte ett |\fothär|,|\fothar| kommando finns där också.

\paragraph{Karaktärsbeskrivningar}
\begin{lstlisting}
\begin{rollbeskrivning}
	\namn{...}
	\bild{\includegraphics[height=\rollbildhojd]{...}}
	\beskrivning{...}
\end{rollbeskrivning}
\end{lstlisting}
Om man vill ha rollbeskrivningar med bilder så kan man använda rollbeskrivningsmiljön för att skapa dem.


\section{Slut på manus}
\label{sec:slut-på-manus}
\index{Listor}
\index{Slut}
Efter sista akten kan man tänkas vilja ha med lite finurliga listor som berättar saker om manuset.

\paragraph{\tt\bfseries\textbackslash kuplettlista}
\index{Kuplettlista}
\index{kuplettlista@\verb#\kuplettlista#}
\addcontentsline{toc}{subsection}{Kuplettlista}
Skriver ut en lista över alla kupletter som finns nämnda så långt. Denna lista är samma som sammanfattningslistan fast med endast kupletterna inkluderade.

\paragraph{\tt\bfseries\textbackslash musiklista}
\index{Musiklista}
\index{musiklista@\verb#\musiklista#}
\addcontentsline{toc}{subsection}{Musiklista}
Skriver ut en lista över alla musikstycken som finns nämnda så långt. Denna lista är samma som sammanfattningslistan fast med endast musik.

\paragraph{\tt\bfseries\textbackslash melodilista}
\index{Melodilista}
\index{melodilista@\verb#\melodilista#}
\addcontentsline{toc}{subsection}{Melodilista}
Skriver ut en lista över alla kupletter och melodier. Specialversion av |\kuplettlista|.

\paragraph{\tt\bfseries\textbackslash rekvisitalista}
\index{Rekvisitalista}
\index{rekvisitalista@\verb#\rekvisitalista#}
\addcontentsline{toc}{subsection}{Rekvisitalista}
Skriver ut en lista över all rekvisita som finns nämnd så långt. Listan är indelad efter akt. Rekvisitan är sorterad i den ordning den först dyker upp.

\paragraph{\tt\bfseries\textbackslash ljudlista}
\index{Ljudlista}
\index{ljudlista@\verb#\ljudlista#}
\addcontentsline{toc}{subsection}{Ljudlista}
Skriver ut en lista över alla ljud som finns nämnda så långt. Listan är indelad efter akt. 

\paragraph{\tt\bfseries\textbackslash sammanfattningslista}
\index{Sammanfattningslista}
\index{sammanfattningslista@\verb#\sammanfattningslista#}
\addcontentsline{toc}{subsection}{Sammanfattningslista}
Skriver ut en sammanfattning som består utav de sammanfattningar som angetts till varje scen samt de kupletter som finns.

\paragraph{\tt\bfseries\textbackslash todolista}
\index{TODO-lista}
\index{todolista@\verb#\todolista#}
\addcontentsline{toc}{subsection}{TODO-lista}
Skriver ut en lista över TODO-noteringar.

\paragraph{\tt\bfseries\textbackslash kommentarlista}
\index{Kommentarslista}
\index{kommentarlista@\verb#\kommentarlista#}
\addcontentsline{toc}{subsection}{Kommentarslista}
Skriver ut en lista över kommentarer.

\subsection*{Statistik}
\paragraph{\tt\bfseries\textbackslash statistiktabell}
\index{Statistiktabell}
\index{statistiktabell@\verb#\statistiktabell#}
\addcontentsline{toc}{subsection}{Statistiktabell}
\label{sec:statistik}
Skriver ut statistik för varje definierad karaktär men inte för hjälpkaraktärer, dvs de som skapats via |\nyroll| men inte via |\nyhjälproll|.

Inkluderat i statistiken är
\begin{description}
	\item[replikantal] --- antal |\Namn|-kommandon,
	\item[agerandeantal] --- antal |\Namngr|-kommandon,
	\item[sångantal] --- antal |\Namnsj|-kommandon,
	\item[kupletter/musik] --- antal kupletter/musiknummer \emph{då karaktären är på scen}(notera att detta inte har någon koppling till de deltagare som anges via |\deltagare|),
	\item[närvaro] --- antal repliker då karaktären är på scen och ur subspel,
	\item[subspel] --- antal repliker då karaktären är på scen i subspel
\end{description}
samt några totaler och även längduppskattning av några av ovanstående värden. Dvs det finns för repliker, närvaro och subspel även uträknat ungeför hur många centimeter papper dessa värden motsvarar i manus. Dessa värden är avrundade men bör vara hyggligt exakta.

För kupletter/musiknummer finns två värden: vanlig räkning av antalet kupletter/musiknummer som varje karaktär deltar i, samt en ``balanserad'' räkning där varje kuplett/musiknummer bidrar med $1/n$ till varje deltagare, där $n$ är antalet deltagare i kupletten/musiknumret.

Då klassalternativet |grafer| är angett kommer även en del grafer inkluderas.



%\section{Exempel}



\appendix



\newpage
\section{Fördefinerade färger}
\index{Färger!Fördefinerade}
\label{app:farger}
\parbox{\textwidth}{\raggedright%
\fboxsep=0pt%
\newcommand{\C}[1]{\makebox[0.3\textwidth][l]{%
	\fbox{\color{#1}\rule{1em}{1em}}\ %
	\textcolor{#1}{\textbf{#1}}%
}}
\C{GreenYellow}
\C{Yellow}
\C{Goldenrod}
\C{Dandelion}
\C{Apricot}
\C{Peach}
\C{Melon}
\C{YellowOrange}
\C{Orange}
\C{BurntOrange}
\C{Bittersweet}
\C{RedOrange}
\C{Mahogany}
\C{Maroon}
\C{BrickRed}
\C{Red}
\C{OrangeRed}
\C{RubineRed}
\C{WildStrawberry}
\C{Salmon}
\C{CarnationPink}
\C{Magenta}
\C{VioletRed}
\C{Rhodamine}
\C{Mulberry}
\C{RedViolet}
\C{Fuchsia}
\C{Lavender}
\C{Thistle}
\C{Orchid}
\C{DarkOrchid}
\C{Purple}
\C{Plum}
\C{Violet}
\C{RoyalPurple}
\C{BlueViolet}
\C{Periwinkle}
\C{CadetBlue}
\C{CornflowerBlue}
\C{MidnightBlue}
\C{NavyBlue}
\C{RoyalBlue}
\C{Blue}
\C{Cerulean}
\C{Cyan}
\C{ProcessBlue}
\C{SkyBlue}
\C{Turquoise}
\C{TealBlue}
\C{Aquamarine}
\C{BlueGreen}
\C{Emerald}
\C{JungleGreen}
\C{SeaGreen}
\C{Green}
\C{ForestGreen}
\C{PineGreen}
\C{LimeGreen}
\C{YellowGreen}
\C{SpringGreen}
\C{OliveGreen}
\C{RawSienna}
\C{Sepia}
\C{Brown}
\C{Tan}
\C{Gray}
\C{Black}
\fbox{\color{white}\rule{1em}{1em}}\ \textbf{(White)}
}



\newpage
\section{Överkurs}

\subsection{Längder}
\index{Längder}
Manusklassen definierar en del längder för att kontrollera utseendet och som kan ändras för att justera detta utseende.

\vspace{1em}
\noindent
\begingroup\small
\begin{tabular}{p{0.36\textwidth}p{0.09\textwidth}p{0.45\textwidth}}
\bf Längd & \bf Default & \bf Beskrivning \\
\hline
|\manus@numbredd| & 3em & Bredden på kolonnen för radnummer. \\
\hline
|\manus@namnbredd| & --- & Bredden på namnkolonnen. Sätts per default automatiskt till att vara så stor så att det längsta rollnamnet passar. \\
\hline
|\manus@indbredd| & 10pt & Bredden på varje närvaroindikator. \\
\hline
|\manus@radavstand| & 5pt & Extra avstånd mellan varje rad. \\
\hline
|\manus@kolavstand| & 1em & Avstånd mellan replikraders kolonner. \\
\hline
|\manus@indkant| & 2pt & Kantbredd på sub\-spels\-indikatorerna. \\
\hline
|\manus@randsep| & 10pt & Avstånd mellan ränderna i sub\-spels\-indi\-kat\-orerna. \\
\hline
|\manus@randbredd| & 5pt & Bredd på ränderna i sub\-spels\-indikat\-orerna. \\
\hline
|\manus@kuplettmellanrum| & 2em & Tomrum före och efter kupletter. \\
\hline
|\manus@musikmellanrum| & 2em & Tomrum före och efter musiknummer. \\
\end{tabular}
\endgroup

%\paragraph{Omdefiniera färg för kupletter}

%\paragraph{Omdefiniera hur \citat{gör-text} ser ut.}

%\paragraph{Rollnamn som innehåller åäö}

\subsection{Mer om sidfötter}
\index{Sidfötter}
\index{fotstilyttre@\texttt{\textbackslash fotstil\at yttre}}
Om man definierar om |\fotstil| till |\fbox|, |\underline| eller liknande så finner man att radbrytningar i sidfötterna slutar fungera som önskat. Detta är för att kommandon som dessa inte är till för paragrafer och inte stödjer radbrytningar.

För att formattera sidfötter på dessa sätt finns istället kommandot \\|\fotstil@yttre|. Detta kommandot appliceras på varje \emph{rad} i sidfötterna istället för på hela paragrafen. Därmed kan man definera om kommandot som till exempel
\begin{lstlisting}
\let\fotstil@yttre\underline
\end{lstlisting}
och få önskat resultat, dvs att varje rad är understruken var för sig istället för hela paragrafen på en lång rad.

Manusklassens definition av |fotstil@yttre| ser ut så här:
\begin{lstlisting}
\def\fotstil@yttre#1{#1}
\end{lstlisting}
vilket innebär att kommandot i normala fall inte gör någonting utan bara skriver ut varje rad som den är.



\printindex



\end{document}
