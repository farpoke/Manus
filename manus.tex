\documentclass[a4paper,12pt]{article}

\usepackage{ifxetex}
\ifxetex
	\RequirePackage[T1]{fontenc}
\else
	\RequirePackage[utf8]{inputenc}
\fi

\usepackage[pdftex, margin=3cm]{geometry}

\usepackage{xifthen}
\usepackage{url}
\usepackage[bottom,hang]{footmisc}
\usepackage[babel]{csquotes}
\usepackage[british,swedish]{babel}
\usepackage[colorlinks]{hyperref}
\usepackage{datetime}
\usepackage[dvipsnames]{xcolor}

\usepackage{fancyvrb}
\DefineShortVerb{\|}

\usepackage{listings}
\lstset{
	basicstyle=\small\ttfamily,
	columns=flexible,
	numberstyle=\tiny,
	language=TeX,
	keywordstyle=\bfseries,
	morekeywords={documentclass,definecolor,nyroll,nyhjalproll,begin,gr,rekv,akt,scen,allgr,titel,deltagare,beskrivning,melodi,Namn,Namngr,Namnin,Namnut,Namnsub,Namnsubslut,rekvisitalista,kuplettlista,sammanfattningslista,statistiktabell,namnbyte,fotfil,pagestyle,thispagestyle,textit,fotstil,fotstil@yttre,textcolor,rule}
}

\renewcommand\ttdefault{pcr}

\usepackage{makeidx}
\def\at{@}
\makeindex

\usepackage{needspace}

\setcounter{secnumdepth}{1} 

\newcommand*{\pack}{\textsf}

\renewcommand{\dateseparator}{-}
\newcommand{\todayiso}{%
	\the\year\dateseparator\twodigit\month\dateseparator\twodigit\day}

\newcommand\funnybone[1]{%
	\raisebox{1em}[1em][0pt]{%
		\makebox[\textwidth][r]{\large\it eller, ``#1''}%
	}%
}



\begin{document}

\title{Att skriva manus med \pack{manus.cls}}
\author{%
	Cecilia Kjellman, \texttt{cecilia.kjellman@gmail.com}\\%
	Anton Mårtensson, \texttt{anton.v.martensson@gmail.com}}
\date{\todayiso}
\maketitle



\newpage
\tableofcontents
\newpage



\section{Vad är \pack{manus.cls}?}

\subsection{Kort svar}
En dokumentklass till \LaTeX\ som är skriven med spex och andra scenframträdanden i tankarna.

\subsection{Något längre svar}
Klassen ger tillgång till ett antal kommandon som tillsammans ger möjlighet att skriva ett snyggt och lättläst manus. Vissa av dessa ger möjlighet till sådana flashiga funktioner som att skriva färgkodade repliker, indikatorer på vilka som är på scen, listor över rekvisita/kupletter/sammanfattningar, numrerade repliker och statistik för respektive roll. Andra är mer av estetisk natur, så som |\akt|. 

\subsection{Kan jag använda klassen trots att jag inte kan \LaTeX?}
Du behöver veta hur man kompilerar filer samt förstå syntaxen som presenteras i detta dokument. Ifall något skulle gå sönder så underlättar det om du har grundläggande felsökningserfarenhet.



\section{Från början: Klassinställningar och preamble}
Låt oss börja med en kort exempel-preamble:

\begin{lstlisting}
\documentclass{manus}
	% <-- Ev. laddning av fler paket.
	% Så som inputenc eller fontenc.
\definecolor{rosa}{rgb}{1,0.5,0.5}
\nyroll[rosa]{Bobby}
\end{lstlisting}

\noindent
Nu är du redo att skriva ett manus om Bobby som kommer få alla sina repliker i rosa! Men låt oss först titta närmare på vad som faktiskt står ovan.

\subsection{Ladda klassen}
\index{documentclass@{\verb#\documentclass{manus}#}}
På rad 1 ovan så anges att önskad dokumentklass är \pack{manus}, vilken bygger på dokumentklassen \pack{article}.

För att ge direktiv till manusklassen kan extra klassalternativ anges på följande sätt:

\begin{lstlisting}
\documentclass[alternativlista]{manus}
\end{lstlisting}

\noindent
där |alternativlista| är en komma-separerad lista med alternativ (eller tom för den delen).

\subsubsection{Alternativ: \texttt{svartvit}}
\label{alt:svartvit}
\index{svartvit@\texttt{svartvit}}
Som standard är alla repliker samt kupletter markerade med färg. Kupletter med en egen kuplettfärg, och repliker efter vilken roll som skall framföra dem. Detta kanske inte är önskvärt om man skall skriva ut ett utkast på en skrivare utan färgkassett. Genom att då lägga till detta enda ord i källfilen kan man producera en svartvit version och lätt ändra tillbaka för när man skall fortsätta arbeta i färg på skärmen.

\subsubsection{Alternativ: \texttt{sammanfattningar}}
\label{alt:sammanfattningar}
\index{sammanfattningar@\texttt{sammanfattningar}}
När ny scen påbörjas med kommandot |\scen| (se nedan) kan en sammanfattning av scenen anges. I normala fall sparas den bara undan för att skrivas ut i slutet av dokumentet i sammanfattningslistan (se nedan). Men om detta klassalternativ ges så skrivs sammanfattningen även ut direkt efter scenrubriken.

\subsubsection{Alternativ: \texttt{grafer}}
\label{alt:grafer}
\index{grafer@\texttt{grafer}}
Detta alternativ gör att statistiktabellen (se avsnitt \ref{sec:statistik}) följs av diverse illustrativa grafer. Denna funktionalitet är ännu experimentell, men fungerande.

\subsubsection{Alternativ: \texttt{todo}}
\label{alt:todo}
\index{todo@\texttt{todo}}
Detta alternativ gör att kommentarer skrivna med kommandot |\todo{...}| visas i texten. En lista över kommentarer kan inkluderas med kommandot |\todolista|. Listan fungerar även om |todo|-alternativet inte anges.

\subsubsection{Alternativ: \texttt{kommentarer}}
\label{alt:kommentarer}
\index{kommentarer@\texttt{kommentarer}}
Gör samma sak som |todo| men med kommandot |\kommentar|, samt att utseendet är något annorlunda. En lista över kommentarer kan inkluderas med \\|\kommentarlista|.

\subsubsection{Alternativ: \texttt{kallrad}}
\label{alt:kallrad}
\index{kallrad@\texttt{kallrad}}
Gör att källradsnumret (dvs den rad i |.tex|-filen där repliken är skriven) visas till vänster om varje repliknumer.

\subsection{Definiera karaktärer/roller}
\label{sec:nyroll}
\index{Definiera karaktärer}
\index{nyroll@\verb#\nyroll#}
\index{nyhjalproll@\verb#\nyhjalproll#}
\funnybone{Våra tävlande är...}
För att kunna använda manusfunktionaliteten behöver man definera vilka karaktärer som finns. Detta görs med två olika kommandon:

\begin{lstlisting}
\nyroll[färg]{namn}      % Ny karaktär.
\nyhjalproll[färg]{namn} % Ny hjälpkaraktär.
\end{lstlisting}

De första varianten definierar en ny karaktär och ger även denna en närvaroindikator som ritas ut i högermarginalen. Mer om denna indikator i avsnitt \ref{sec:narvaro}.

Den andra varianten definierar också en ny karaktär, men skapar \emph{inte} en närvaroindikator eller de kommandon som associeras med en sådan. Detta är användbart för småroller som till exempel en statist som bara finns med i en scen eller en berättare som inte finns på scen.

\subsubsection{Färger}
\label{sec:farger}
\index{Färger}
Parametern |[färg]| är frivillig och om den inkluderas kommer den givna färgen användas för repliker och agerande från denna karaktär samt för karaktärens närvaroindikator och statistik.

Detta är som sagt frivilligt, och svart kommer tilldelas som karaktärsfärg om inget annat anges. Undertecknad tycker dock att svart är en tråkig färg.

Värdet på färgparametern skall vara en färg enligt paketet \pack{xcolor}, som är det paket som \pack{manus.cls} använder för färgläggning. För enkel användning finns ett antal fördefinierade färger som listas i appendix \ref{app:farger}, men man kan även definiera egna färger eller ange en blandning av färger enligt \pack{xcolor}:s format. För detaljer se \pack{xcolor}:s dokumentation, som enklast fås genom att köra ``|texdoc xcolor|'' i en lämplig terminal (en med korrekt \TeX-installation) eller genom att googla ``xcolor''.



\needspace{5em}
\section{Manus}
Början på ett exempelmanus:

\begin{lstlisting}
\begin{document}
	% <-- Ev. andra sidor så som titelsida, innehåll, etc.
\akt
\scen{Bobby pratar}{
	Bobby pratar lite, gör något och pratar lite till.
}
\Bobby{Detta säger jag, och se här \gr{gör något} vad jag gör.}
\Bobbygr{Bobby gör något.}
\Bobby{Här säger jag något igen.}
\end{lstlisting}



\subsection{Akter och scener}
\index{akt@\verb#\akt#}
\index{scen@\verb#\scen#}
\funnybone{Nu får scengruppen jobba!}
Ett manus är indelat i akter och scener, och dessa inleds med följande två kommandon:

\begin{lstlisting}
\akt[Första scennummer]
\scen{Titel}{Sammanfattning}
\end{lstlisting}

\paragraph{Första scennummer} är frivilligt att ange och anger var numreringen av scenerna skall börja. Detta är användbart eftersom vissa akter har en scen 0, andra inte. Om detta inte anges antas att numreringen skall börja på 1.

\paragraph{Titel} är titel för scenen.

\paragraph{Sammanfattning} är sammanfattning för scenen. Denna används för att generera sammanfattning utav hela manuset (se avsnitt \ref{sec:slut-på-manus}). Om klassalternativet \texttt{samman\-fattningar} har angetts skrivs sammanfattningen även ut direkt efter scentiteln.



\subsection{Repliker och agerande}
\label{sec:repliker}
\index{Repliker}
\index{Agerande}
\funnybone{Vems replik är det egentligen?}
Den allmänna formen för att skriva repliker är:
\index{Namn@\verb#\Namn#}
\index{Namngr@\verb#\Namngr#}
\begin{lstlisting}
\Namn{...}
\Namngr{...}
\end{lstlisting}
där ``Namn'' är tidigare definierad i |\nyroll| (se \ref{sec:nyroll}) och där |\Namn| skriver ut en vanlig replik från denne karaktären medans |\Namngr| skriver ut ett ``agerande'' från densamme.

Finessen med |\Namngr| är att kunna skriva längre mängd text för att beskriva att någon gör något. För att markera samma sak mitt i en replik används |\gr|.

Ifall flera roller är involverade är |\allgr{}|\index{allgr@\verb#\allgr#} rätt kommando.

Då flera repliker i rad är från samma karaktär kommer namnet endast skrivas ut för den första raden.



\subsection{Scennärvaro}
\label{sec:narvaro}
\index{Namnin@\verb#\Namnin#}
\index{Namnut@\verb#\Namnut#}
\funnybone{Knack, knack. Vem där?}
\funnybone{Var tusan kom han ifrån?}
De karaktärer som definierats med kommandot |\nyroll| får en närvaroindikator i högermarginalen som visar om karaktären är på scen eller inte samt om denna är i subspel eller inte.

Denna indikator kontrolleras med följande fyra kommandon:
\begin{lstlisting}
\Namnin
\Namnut
\Namnsub
\Namnsubslut
\end{lstlisting}

Till exempel: när |\Bobbyin| skrivs kommer Bobby markeras som inne på scen och närvaroindikatorn för Bobby kommer visas i marginalen. När Bobby sedan skall av scen skriver man |\Bobbyut| och närvaroindikatorn avslutas.

Om Bobby skall vara på scen men i subspel skrivs |\Bobbysub|, vilket gör att Bobbys närvaroindikator blir streckad. När Bobby är klar med vad han än höll på med och återigen skall agera stort skrivs |\Bobbysubslut|.

|\Namnin|- och |\Namnut|-kommandona skriver även ut ett automatiskt agerande som meddelar att karaktären i fråga ``kommer in på scen'' respektive ``går av scen.'' Då detta ej önskas läggs en asterisk till efter kommandot (till exempel |\Bobbyin*|).

Det spelar ingen roll om man inte skriver |\Namnsubslut| innan |\Namnut| (dvs karaktären i fråga går av scen direkt från subspel) men |\Namnin| kommer automatiskt ta karaktären ut ur subspel.



\subsection{Kupletter}
\index{Kuplett}
\funnybone{Allsång på Scenen}
Kupletter anges på formen:
\begin{lstlisting}
\begin{kuplett}[kommandon]
	\titel{...}
	\beskrivning{...}
	\melodi{...}
\end{kuplett}
\end{lstlisting}
där samtliga kuplettkommandon (|\titel| osv...) är frivilliga. Detta skriver ut en ny kuplett i manus och lägger även till samma kuplett i innehållsförteckningen och sammanfattningslistan.

Alla karaktärer som är på scen (inklusive subspel) när kupletten skrivs antas vara deltagare och deras namn kommer listas som sådana.

Parametern |[kommandon]| är en frivillig parameter som kan användas för att ange extra kommandon som skall gälla innuti kuplettmiljön. Detta är främst tänkt att användas för att enkelt justera vilka som skall vara på scen under kupletten.

Om till exempel Bobby och Eva är av scen men skall in på scen under en kuplett kan man specificera:
\needspace{3\baselineskip}
\begin{lstlisting}
\begin{kuplett}[\Bobbyin*\Evain*]
% ...
\end{kuplett}
\end{lstlisting}
Detta gör att Bobby och Eva kommer in under kupletten och markeras som deltagare, och sedan automatiskt går av scen efter kupletten.

\paragraph{Obs} Det finns ingen speciell begränsning på vad som får skrivas inuti kuplett\-kommandona, men det finns risk att andra delar av kuplettfunktionaliteten (speciellt sammanfattningarna) går sönder då underligare kommandon används inuti till exempel beskrivningen.



\section{Övriga hjälpmedel}
\funnybone{Ingenjörsmässiga lösningar på problem}
För att hjälpa manusskrivanded finns även några extra kommandon:

\paragraph{\tt\bfseries\textbackslash berattare\{...\}}
\index{Berättare}
\index{berattare@\verb#\berattare#}
Detta kommandot finns för att berätta något som händer som inte angår någon karaktär direkt. Så som hur kulisserna ser ut eller vilken rekvisita som står på scen. Rekvisita som nämns här kommer markeras annorlunda i listan över rekvisita (se avsnitt \ref{sec:slut-på-manus}). Notera även att detta inte är samma sak som att ha en berättar-roll med repliker.

\paragraph{\tt\bfseries\textbackslash comment\{...\}}
\index{comment@\verb#\comment#}
Detta kommando finns ifall man utav någon anledning inte vill kommentera ut resten av en rad med det på \TeX\ sedvanliga sättet med \%. Detta kommando tar helt enkelt bara bort det som befinner sig inom måsvingarna, och kan göra koden mer svårläst.

\paragraph{\tt\bfseries\textbackslash rekv\{...\}}
\index{Rekvisita}
\index{rekv@\verb#\rekv#}
När rekvisita refereras till i manus skall den markeras med detta kommando. Detta markerar rekvisita-texten och indexerar även densamma i sakregistret.

\paragraph{\tt\bfseries\textbackslash TODO\{...\}}
\index{TODO}
\index{todo@\verb#\todo#}
Detta kommando lägger till en TODO-notering med angiven text. Då klassalternativet |todo| är angivet kommer denna notering skrivas ut direkt vid anrop, annars inte. I båda fallen läggs det till i TODO-listan som kan skrivas ut med kommandot |\todolista|.

\paragraph{\tt\bfseries\textbackslash kommentar\{...\}}
\index{kommentar}
\index{kommentar@\verb#\kommentar#}
Detta kommando lägger till en kommentar på liknande sätt som |\TODO| ovan, med skillnaden att klassalternativet |kommentarer| styr, och att listan över kommentarer ges med |\kommentarlista|.

\subsection{Definiera roller, episod II}
\index{Definiera karaktärer}
\index{nyroll@\verb#\nyroll#}
\index{nyhjalproll@\verb#\nyhjalproll#}
\funnybone{Inte bara för George Lucas}
I avsnitt \ref{sec:nyroll} beskrevs kommandon för att definiera karaktärer i manuset. Dessa var dock inte helt kompletta. Hur skall man till exempel göra då karaktärens namn innehåller ett å, ä eller ö? Eller om rollen skall byta namn under pågående manus?

De faktiska kommandona som definierar roller ser ut så här:

\begin{lstlisting}
\nyroll[färg]{namn-a}[namn-b]      % Ny karaktär.
\nyhjalproll[färg]{namn-a}[namn-b] % Ny hjälpkaraktär.
\end{lstlisting}

\noindent
där parametrarna |[färg]| och |{namn-a}| motsvarar de tidigare givna. 

Skillnaderna mellan |namn-a| och |namn-b| är att det första är det namn som används för kommandon, medans det andra är det namn som faktiskt skrivs ut i manus. |[namn-b]| är en frivillig parameter och om den utelämnas kommer |namn-a| användas.

Så, om vi till exempel vill ha en berättarröst som karaktär (inte samma sak som |\berattare| ovan) kan vi göra så här:

\begin{lstlisting}
\nyroll{Berattare}[Berättare]
% ... manus ...
\Berattare{Här säger berättaren någonting.}
\end{lstlisting}

\vspace{-1em}
\paragraph{Namnbyte} En annan sak som nämndes var namnbyte, som åstakomms med kommandot |\namnbyte{namn-a}{namn-b}| som ändrar det namn som skrivs ut (|namn-b|) för karaktären med kommandonamn |namn-a|. Detta kan vara användbart då man, så som händer i spex, tar död på en karaktär och skådespelaren skall fortsätta i en ny karaktär med samma färg och indikator.

\subsection{Sidfötter}
\index{Sidfötter}
\index{fotfil@\verb#\fotfil#}
\funnybone{Fotfil, smakar som ostbågar}
Manusklassen innehåller ett kommando för att specificera en fil med sidfötter som skall användas:

\begin{lstlisting}
\fotfil[layoutnamn]{filnamn}
\end{lstlisting}

Detta kommando skapar en ny sidlayout som för varje sida läser in en rad från den angivna filen och använder den raden som sidfot för den aktuella sidan. Tomrader i filen resulterar i sidor utan sidfotstext (en sida per rad), dock fortfarande med sidnummer på rätt ställe.

Parametern |[layoutnamn]| är frivillig och namnet |fotfil| kommer användas om den utelämnas. Detta namn är det namn som ges den nya sidlayouten och ett fel kommer ges om en sidlayout med det namnet redan finns. Den nya sidlayouten aktiveras sedan direkt.

\paragraph{Layoutändring} Sidlayouter som skapas via detta kommando kan precis som \\\LaTeX:s vanliga sidlayouter styras via kommandona |\pagestyle| och \\|\thispagestyle|. Dvs om man vill tillfälligt ändra layout kan man till exempel använda kommandot
\begin{lstlisting}
\thispagestyle{plain}
\end{lstlisting}
som för just denna sidan ändrar layouten till |plain| (\LaTeX:s defaultlayout).

Om man vill helt gå tillbaka till vanligare layouter kan man exempelvis använda
\begin{lstlisting}
\pagestyle{plain}
\end{lstlisting}
som ändrar layouten för följande sidor. Om man senare vill ändra tillbaka kan man använda samma kommando men ange en layout som skapats via |\fotfil|. Om layoutnamnet inte angavs explicit så användes |fotfil| som namn, och kommandot för att ändra tillbaka blir då
\begin{lstlisting}[keywords={pagestyle}]
\pagestyle{fotfil}
\end{lstlisting}

Kommandot |\fotfil| kan dessutom användas flera gånger för att skapa olika sidlayouter som läser från olika filer, och man kan sedan fritt växla mellan dem med hjälp av ovan nämnda kommando.

\paragraph{Utseende}\index{fotstil@\verb#\fotstil#} Utseendet på sidfötterna styrs främst av kommandot |\fotstil|. Detta kommando kan defineras till att ta noll eller en parameter och används innan sidfotstexten skrivs ut.

Manusklassens definition av |\fotstil| ser ut så här:
\begin{lstlisting}
\let\fotfil\textit
\end{lstlisting}
vilket gör att sidfotstexten skrivs ut i kursiv stil.

Andra exempel är:
\begin{lstlisting}
\let\fotstil\bf % Fetstil.
\def\fotstil#1{\textcolor{green}{#1}} % Grön text.
\def\fotstil{\rule{5pt}{5pt}} % En liten låda innan texten.
\end{lstlisting}

\paragraph{Finkontroll}\index{forhar@\verb#\fothar#} I vissa fall kan det vara önskvärt att sätta en specifik fotnot på en sida. Istället för att pilla med fot-filen så att rätt sak hamnar på rätt sida finns istället kommandot

\begin{lstlisting}
\fothar{...}
\end{lstlisting}

Detta kommando gör att den angivna texten används som fotnot på den nuvarande sidan, istället för texten från fot-filen. Normala fotnötter kommer sedan fortsätta på nästa sida, såvida inte ett |\fothar| kommando finns där också.



\section{Slut på manus}
\label{sec:slut-på-manus}
\index{Listor}
\index{Slut}
\funnybone{Manusgruppen somnade}
Efter sista akten kan man tänkas vilja ha med lite finurliga listor som berättar saker om manuset.
\begin{lstlisting}
\rekvisitalista
\kuplettlista
\sammanfattningslista
\statistiktabell
\end{document}
\end{lstlisting}

\paragraph{\tt\bfseries\textbackslash kuplettlista}
\index{Kuplettlista}
\index{kuplettlista@\verb#\kuplettlista#}
\addcontentsline{toc}{subsection}{Kuplettlista}
Skriver ut en lista över alla kupletter som finns nämnda så långt. Denna lista är samma som sammanfattningslistan fast med endast kupletterna inkluderade.

\paragraph{\tt\bfseries\textbackslash melodilista}
\index{Melodilista}
\index{melodilista@\verb#\melodilista#}
\addcontentsline{toc}{subsection}{Melodilista}
Skriver ut en lista över alla kupletter och melodier. Specialversion av |\kuplettlista|.

\paragraph{\tt\bfseries\textbackslash rekvisitalista}
\index{Rekvisitalista}
\index{rekvisitalista@\verb#\rekvisitalista#}
\addcontentsline{toc}{subsection}{Rekvisitalista}
Skriver ut en lista över all rekvisita som finns nämnd så långt. Lista är indelad efter akt. Rekvisita som är nämnd i |\berattare| är markerad ty den bör rimligen vara rekvisita som skall ligga framme i början av akten. Rekvisitan är sorterad i den ordning den först dyker upp.

\paragraph{\tt\bfseries\textbackslash sammanfattningslista}
\index{Sammanfattningslista}
\index{sammanfattningslista@\verb#\sammanfattningslista#}
\addcontentsline{toc}{subsection}{Sammanfattningslista}
Skriver ut en sammanfattning som består utav de sammanfattningar som angetts till varje scen samt de kupletter som finns.

\paragraph{\tt\bfseries\textbackslash todolista}
\index{TODO-lista}
\index{todolista@\verb#\todolista#}
\addcontentsline{toc}{subsection}{TODO-lista}
Skriver ut en lista över TODO-noteringar.

\paragraph{\tt\bfseries\textbackslash kommentarlista}
\index{Kommentarslista}
\index{kommentarlista@\verb#\kommentarlista#}
\addcontentsline{toc}{subsection}{Kommentarslista}
Skriver ut en lista över kommentarer.

\subsection*{Statistik}
\paragraph{\tt\bfseries\textbackslash statistiktabell}
\index{Statistiktabell}
\index{statistiktabell@\verb#\statistiktabell#}
\addcontentsline{toc}{subsection}{Statistiktabell}
\label{sec:statistik}
Skriver ut statistik för varje definierad karaktär men inte för hjälpkaraktärer, dvs de som skapats via |\nyroll| men inte via |\nyhjalproll|.

Inkluderat i statistiken är
\begin{description}
	\item[replikantal] --- antal |\Namn|-kommandon,
	\item[agerandeantal] --- antal |\Namngr|-kommandon,
	\item[kupletter] --- antal kupletter \emph{då karaktären är på scen}(notera att detta inte har någon koppling till de deltagare som anges i kupletten via |\deltagare|),
	\item[närvaro] --- antal repliker då karaktären är på scen och ur subspel,
	\item[subspel] --- antal repliker då karaktären är på scen i subspel
\end{description}
samt några totaler och även längduppskattning av några av ovanstående värden. Dvs det finns för repliker, närvaro och subspel även uträknat ungeför hur många centimeter papper dessa värden motsvarar i manus. Dessa värden är avrundade men bör vara hyggligt exakta.

För kupletter finns två värden: vanlig räkning av antalet kupletter som varje karaktär deltar i, samt en ``balanserad'' räkning där varje kuplett bidrar med $1/n$ till varje deltagare, där $n$ är antalet deltagare i kupletten.

Då klassalternativet |grafer| är angett kommer även en del grafer inkluderas.



%\section{Exempel}



\appendix



\newpage
\section{Fördefinerade färger}
\index{Färger!Fördefinerade}
\label{app:farger}
\parbox{\textwidth}{\raggedright%
\fboxsep=0pt%
\renewcommand{\C}[1]{\makebox[0.3\textwidth][l]{%
	\fbox{\color{#1}\rule{1em}{1em}}\ %
	\textcolor{#1}{\textbf{#1}}%
}}
\C{GreenYellow}
\C{Yellow}
\C{Goldenrod}
\C{Dandelion}
\C{Apricot}
\C{Peach}
\C{Melon}
\C{YellowOrange}
\C{Orange}
\C{BurntOrange}
\C{Bittersweet}
\C{RedOrange}
\C{Mahogany}
\C{Maroon}
\C{BrickRed}
\C{Red}
\C{OrangeRed}
\C{RubineRed}
\C{WildStrawberry}
\C{Salmon}
\C{CarnationPink}
\C{Magenta}
\C{VioletRed}
\C{Rhodamine}
\C{Mulberry}
\C{RedViolet}
\C{Fuchsia}
\C{Lavender}
\C{Thistle}
\C{Orchid}
\C{DarkOrchid}
\C{Purple}
\C{Plum}
\C{Violet}
\C{RoyalPurple}
\C{BlueViolet}
\C{Periwinkle}
\C{CadetBlue}
\C{CornflowerBlue}
\C{MidnightBlue}
\C{NavyBlue}
\C{RoyalBlue}
\C{Blue}
\C{Cerulean}
\C{Cyan}
\C{ProcessBlue}
\C{SkyBlue}
\C{Turquoise}
\C{TealBlue}
\C{Aquamarine}
\C{BlueGreen}
\C{Emerald}
\C{JungleGreen}
\C{SeaGreen}
\C{Green}
\C{ForestGreen}
\C{PineGreen}
\C{LimeGreen}
\C{YellowGreen}
\C{SpringGreen}
\C{OliveGreen}
\C{RawSienna}
\C{Sepia}
\C{Brown}
\C{Tan}
\C{Gray}
\C{Black}
\fbox{\color{white}\rule{1em}{1em}}\ \textbf{(White)}
}



\newpage
\section{Överkurs}

\subsection{Längder}
\index{Längder}
Manusklassen definierar en del längder för att kontrollera utseendet och som kan ändras för att justera detta utseende.

\vspace{1em}
\noindent
\begingroup\small
\begin{tabular}{p{0.36\textwidth}p{0.09\textwidth}p{0.45\textwidth}}
\bf Längd & \bf Default & \bf Beskrivning \\
\hline
|\manus@numbredd| & 3em & Bredden på kolonnen för radnummer. \\
\hline
|\manus@namnbredd| & --- & Bredden på namnkolonnen. Sätts per default automatiskt till att vara så stor så att det längsta rollnamnet passar. \\
\hline
|\manus@indbredd| & 10pt & Bredden på varje närvaroindikator. \\
\hline
|\manus@radavstand| & 5pt & Extra avstånd mellan varje rad. \\
\hline
|\manus@kolavstand| & 1em & Avstånd mellan replikraders kolonner. \\
\hline
|\manus@indkant| & 2pt & Kantbredd på sub\-spels\-indikatorerna. \\
\hline
|\manus@randsep| & 10pt & Avstånd mellan ränderna i sub\-spels\-indi\-kat\-orerna. \\
\hline
|\manus@randbredd| & 5pt & Bredd på ränderna i sub\-spels\-indikat\-orerna. \\
\hline
|\manus@kuplettmellanrum| & 2em & Tomrum före och efter kupletter. \\
\end{tabular}
\endgroup

%\paragraph{Omdefiniera färg för kupletter}

%\paragraph{Omdefiniera hur \citat{gör-text} ser ut.}

%\paragraph{Rollnamn som innehåller åäö}

\subsection{Mer om sidfötter}
\index{Sidfötter}
\index{fotstilyttre@\texttt{\textbackslash fotstil\at yttre}}
Om man definierar om |\fotstil| till |\fbox|, |\underline| eller liknande så finner man att radbrytningar i sidfötterna slutar fungera som önskat. Detta är för att kommandon som dessa inte är till för paragrafer och inte stödjer radbrytningar.

För att formattera sidfötter på dessa sätt finns istället kommandot \\|\fotstil@yttre|. Detta kommandot appliceras på varje \emph{rad} i sidfötterna istället för på hela paragrafen. Därmed kan man definera om kommandot som till exempel
\begin{lstlisting}
\let\fotstil@yttre\underline
\end{lstlisting}
och få önskat resultat, dvs att varje rad är understruken var för sig istället för hela paragrafen på en lång rad.

Manusklassens definition av |fotstil@yttre| ser ut så här:
\begin{lstlisting}
\def\fotstil@yttre#1{#1}
\end{lstlisting}
vilket innebär att kommandot i normala fall inte gör någonting utan bara skriver ut varje rad som den är.



\printindex



\end{document}
